\documentclass[12pt,a4paper]{article}

% Paquetes
\usepackage[utf8]{inputenc}
\usepackage[spanish]{babel}
\usepackage{amsmath}
\usepackage{graphicx}
\usepackage{geometry}
\usepackage{hyperref}
\usepackage{listings}
\usepackage{xcolor}
\usepackage{enumitem}
\usepackage{float}

% Configuración de página
\geometry{margin=2cm}

% Configuración de código
\lstset{
    basicstyle=\ttfamily\small,
    breaklines=true,
    frame=single,
    backgroundcolor=\color{gray!10},
    keywordstyle=\color{blue},
    commentstyle=\color{green!60!black},
    stringstyle=\color{red}
}

% Información del documento
\title{\textbf{Sistema de Seguimiento de Tortugas con ROS2} \\ 
       \large Trabajo 2 - EOII}
\author{Francisco Nortes Novikov \\ Vicente Burdeus Sánchez}
\date{16 de enero de 2026}

\begin{document}

\maketitle

\section{Descripción de la Implementación}

El sistema desarrollado consiste en un seguidor de tortugas en ROS2 que permite que una tortuga exploradora siga automáticamente a otra tortuga controlada manualmente. Se implementó una arquitectura modular con los siguientes componentes:

\textbf{Paquetes creados:}
\begin{itemize}[noitemsep]
    \item \textbf{follower}: Paquete principal con la lógica de seguimiento
    \item \textbf{follower\_interfaces}: Definiciones de servicios y actions personalizados
\end{itemize}

\textbf{Nodos principales:}
\begin{itemize}[noitemsep]
    \item \textbf{pose\_savers}: Suscriptor a \texttt{/turtle1/pose} y \texttt{/explorer/pose}
    \item \textbf{explorer\_velocity}: Controlador proporcional que publica velocidades en \texttt{/explorer/cmd\_vel}
    \item \textbf{turtle\_info\_service}: Servidor del servicio de información
    \item \textbf{catch\_info\_action}: Action server para el proceso de captura
\end{itemize}

\textbf{Ecuaciones del controlador:} $d = \sqrt{(x_t - x_e)^2 + (y_t - y_e)^2}$, $\alpha = \text{atan2}(y_t - y_e, x_t - x_e)$, $\theta_e = \text{atan2}(\sin(\alpha - \theta_{actual}), \cos(\alpha - \theta_{actual}))$, $v = 1.0 \times d \times \cos(\theta_e)$, $\omega = 4.0 \times \theta_e$.

\section{Interfaces Implementados}

\subsection{Servicio TurtleInfo.srv}

Proporciona información completa sobre ambas tortugas:

\textbf{Request:} Vacío

\textbf{Response:} \texttt{float64} para posiciones (\texttt{turtle\_x/y}, \texttt{explorer\_x/y}), orientaciones (\texttt{turtle\_theta}, \texttt{explorer\_theta}), velocidades (\texttt{turtle/explorer\_linear/angular\_velocity}) y \texttt{distance}.

\subsection{Action CatchTurtle.action}

Permite monitorizar el proceso de captura con feedback continuo:

\textbf{Goal:} Vacío

\textbf{Result:} \texttt{bool caught} (indica si la tortuga fue capturada)

\textbf{Feedback:} Los mismos campos que la respuesta del servicio TurtleInfo

\section{Problemas Encontrados y Soluciones}

\subsection{Bloqueo del Action Server}

\textbf{Problema:} Al implementar inicialmente el action server en el nodo principal, este bloqueaba la ejecución del resto de callbacks, impidiendo que el controlador funcionara correctamente.

\textbf{Solución:} Tras investigar la documentación de ROS2 sobre executors y callback groups, se optó por una arquitectura modular completamente separada. Se creó un nodo independiente para cada funcionalidad principal, utilizando \texttt{SharedPoses} con \texttt{threading.Lock} para compartir datos de forma thread-safe entre nodos.

\subsection{Sincronización de Datos}

\textbf{Problema:} Acceso concurrente a las poses desde múltiples nodos.

\textbf{Solución:} Implementación de la clase \texttt{SharedPoses} con locks y \texttt{deepcopy} para garantizar acceso thread-safe a los datos compartidos.

\section{Resultados de Pruebas}

Se realizaron las siguientes pruebas con resultados exitosos:

\begin{enumerate}[noitemsep]
    \item \textbf{Seguimiento básico:} La tortuga exploradora sigue correctamente a turtle1 controlada por teleop
    \item \textbf{Servicio de información:} El cliente recibe y muestra correctamente todos los datos de posición y velocidad
    \item \textbf{Action server:} Proporciona feedback continuo durante el proceso de captura hasta alcanzar el objetivo
\end{enumerate}

El sistema funciona de manera fluida y precisa. Las siguientes capturas muestran los clientes funcionando:

\begin{figure}[H]
    \centering
    \includegraphics[width=0.48\textwidth]{server_client.png}
    \hfill
    \includegraphics[width=0.48\textwidth]{action_client.png}
    \caption{Cliente del servicio de información (izq.) y del action server (der.)}
\end{figure}

\section{Mapa de Nodos, Topics y Servicios}

\begin{figure}[H]
    \centering
    \includegraphics[width=0.75\textwidth]{rosgraph.png}
    \caption{Grafo de nodos del sistema (rqt\_graph)}
\end{figure}

\end{document}